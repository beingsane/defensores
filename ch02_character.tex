\chapter{O Personagem}
\label{ch:character}

RPG é um jogo de interpretação de personagens, então sendo assim, ter personagens para jogar é algo essencial! Sendo assim, esse capítulo é um resumo de tudo que você precisa para construir e entender um personagem de Defensores.

\section{Quem deve criar os personagens?}

Sem sobra de dúvida, a pessoa que criara mais personagens será o Mestre, já que ele controla toda a trama que os jogadores estarão participando. Cada um dos outros jogadores criam o personagem que irão controlar. Eventualmente quando o Mestre é experiente e vai mestrar para pessoas iniciantes que só querem conhecer, o próprio Mestre cria os personagens que o os jogadores irão controlar. 

Exemplo: João decide ser o Mestre, porque ele tem uma história bem legal preparada, leu e entendeu muito bem o manual. André, Luíza e Rafael serão os jogadores, cada um deles deve criar um personagem para participar da história do João.

\section{A criação do personagem começa...}

Com a história! Primeiro é importante saber o cenário e os limites do mundo da história do Mestre, se cenário for fantasia medieval onde existem feiticeiros e dragões seu personagem pode ser de um jeito, agora se o cenário for baseado histórias futuristas onde há lutas de sabre de luz e armas laser, seu personagem será de outra maneira. Enfim, seu personagem tem que ser factível com o mundo da história do Mestre.

Agora que você compreendeu o mundo em que seu personagem viverá, agora é a vez de pensar na história do seu personagem. Onde ele nasceu? Quem eram seus pais? Possui irmãos? Qual o nome do seu personagem? Como foi a infância dele? Que habilidades ele vai ter? Com quem aprendeu as suas habilidades? Ele é sério? Bem humorado? Porque se decidiu aventurar? Como é a sua aparência (cabelos, roupas, cor dos olhos, pele)? Como ele luta? Enfim, essas são algumas questões que podem ajudar a construir seu personagem. Mas não tente pensar em tudo agora, provavelmente você terá uma ideia básica dele agora e depois das distribuições de pontos, você pode querer mudar vários aspectos(ou mudar tudo mesmo) da história para se adequar a ficha do personagem. Por falar em ficha...

Exemplo: O Mestre João diz aos seus jogadores que a aventura se passará no país de Hoveran, mundo que eles vão jogar é de fantasia medieval onde é dragões e magos não são lenda. O jogador André decide que seu personagem será um mago orfão que passou a vida inteira na academia de treinamento dos magos, Luíza diz que sua personagem é uma guerreira amazona, sua família é a tribo onde foi criada e vive nas florestas de Hoveran, Rafael decide que seu personagem é um veterando de guerra do Vietnã, que anda com uma metralhadora e um cinto de granadas. João alerta Rafael que esse tipo de personagem não faz o menor sentido no mundo que irão jogar, pois lá não existe Vietnã e nem metralhadoras ou granada. Rafael, pensa um pouco e sugere que o personagem dele é um infiltrador, que utiliza facas e furtividade para abater seus inimigos a distância.

\section{A ficha de personagem}

A ficha de personagem é o local onde você anotará a construção do seu personagem. Ela pode ser anotações em uma folha de caderno simples ou pode usar essa ficha(url para donwload) para auxiliar a anotação. Nem sempre é necessário preenche-la toda para ter um personagem, muitas vezes você quer anotar tanto detalhes que ela também pode acabar não sendo suficiente. Então, não fique preso a ela!

\section{Pontuação}
Antes de construir o personagem, é importante saber quantos pontos de personagem que inciais que você possui, esses chamaremos de Nível. Assim, um pesonagem de Nível 8(N8), significa que foi construído com 8 pontos de personagem. Esse Nível incial para os jogadores influencia diretamente na construção e até na história do personagem, é determinado pelo Mestre. O natural é que todos os jogadores comecem com o mesmo Nível de personagem, mas por algum motivo especial o Mestre pode decidir níveis diferentes.

As pontuações comuns para personagens são:

\begin{description}

\item[Nível 0-6] Pessoas comuns. A dona de casa, o empresário, o policial, o ladrão, o mecânico, o jornaleiro, o padeiro... enfim, pessoas comuns que não receberam nenhum treinamento para participar de situações de maiores riscos. Não recomendável para pesonagens jogadores.

\item[Nível 8] O aventureiro inciante ou o profissional de elite. São personagens que estão acima da média, ou seja, são pessoas de destaque. Esse é o nível indicado para os aventureiros que acabaram de sair do seu treinamento e estão prontos para buscar suas próprias aventuras.

\item[Nível 14] O aventureiro lendário. São personagens que já sobreviveram muitas aventuras, conquistaram grandes desafios e possuem muita história para contar. Indicado para aventuras de grande risco onde os jogadores precisam ser bastante poderosos para soluciona-los.

\item[Outros níveis] Em casos raros o Mestre pode desejar que os personagens comecem com um nível maior ou menor que 14, não há problemas, o Mestre decide!. Mas o recomendável que no máximo, para personagens recém criados seja N14, para alcançar níveis maiores os jogadores conquistem jogando e obtendo experiência.

\end{description}

\section{As características}

Após determinado o Nível inicial, basta usar o número do nível como pontos e distribuir nas características aprensentadas a seguir. Exemplo: Um personagem de N14, tem 14 pontos para gastar comprando atributos, vantagens, perícias...

\begin{description}

\item[Atributos] Os atributos principais são: Força, Habilidade, Resistência, Mente, Armadura e Poder de Fogo. Os atributos de personagem estão diretamente ligados ao porder de combate. Portanto é essencial que pesonagem dos jogadores tenham bastante pontos gastos aqui, enquanto personagens comuns provavelmente terão quase nenhum ponto gasto em atributos. Cada ponto gasto em atributos equivale a um valor, portanto se o personagem gastou 2 pontos em Força, então ele terá Força 2.

\item[Atributos secundários] Não há necessidade dos jogadores gastar pontos aqui, já que esses atributos são afetados diretamente pelos atributos principais. Entretanto, existem vantagens que aprimoram os atributos secundários. Os atributos secundários são Pontos de Vida, Pontos de Magia, Pontos de Ação, Fadiga, Velocidade, Alcance de Poder de Fogo e Tipo de Dano Especializado.

\item[Raça] Se deseja ser um personagem humano, pode pular essa parte. Caso queira que seu personagem seja um alien, ou um elfo, quem sabe um vampiro ou um androide, terá que "comprar" uma raça nessa sessão. Normalmente os humanos são raças neutras em vários mundos de fantasia, enquanto outras raças apresentam poderes ou até mesmo desvantagens. Portanto, pode ser que ao escolher uma raça desvantajosa você ganhe pontos para gastar com outras coisas, esses pontos entram no limite de pontos máximo de desvantagem no momento da criação do personagem.

\item[Vantagens e Desvantagens] Vantagens são qualquer coisa que traga algum benefício ao personagem durante a aventura, por exemplo, o personagem ter muito dinheiro ou ter um golpe secreto mais forte. As Desvantagens pelo contrário, sempre trazem alguma limitação que prejudique o personagem durante a aventura, em compensação ele ganha mais pontos para gastar. O Mestre e os Jogadores de podem criar mais vantagens e desvantagens além das descritas nesse manual, ou alterar os efeitos delas, desde que todos jogadores concordem com alteração.

\item[Perícias] Nem sempre poderes de combate são essenciais para concluir uma aventura com sucesso, as vezes é necessário saber encontrar comida, encontrar abrigo, hackear computadores, dirigir um carro, pilotar uma avião e etc. Personagens jogadores normalmente não gastam muito pontos com perícias, normalmente um jogador é escolhido para ser o especialista. Personagens que representam pessoas comuns, tem a maioria dos seus pontos de personagem gastos com perícias.

\item[Magias] Magias são uma série de poderes únicos que são normalmente usados por magos, feiticeiros, psiquicos e etc. Quando Magias são adicionadas ao jogo, uma certa complexidade é adicionada devido ao nível de possibilidades que elas permitem ter. Magias não são adquiridas com pontos de personagem, somente as magias inicias básicas através vantagens, as demais magias são conquistadas através das aventuras.

\item[Revisão] Após preencher a ficha de personagem, resta responder: ele está de acordo como imaginou? Precisa mudar algo em sua história para explicar seus atributos, vantagens, desvantagens, perícias e raça? Os pontos estão de acordo com o total de pontos disponíveis? Nesse momento ainda é permitido alguns ajustes, depois da aprovação do Mestre não é mais permitido alterá-lo.

\item[Pertences iniciais] Seu personagem está quase pronto! Só falta decidir o dinheiro e os equipamentos iniciais. Os detalhes sobre esse assunto está no capítulo "A Aventura"!

\end{description} 

\section{Exemplo de Personagem}





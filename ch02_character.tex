\chapter{O Personagem}

RPG é um jogo de interpretação de personagens, então sendo assim, ter personagens para jogar é algo essencial! Sendo assim, esse capítulo é um resumo de tudo que você precisa para construir e entender um personagem de Defensores.

\section{Quem deve criar os personagens?}

Sem sobra de dúvida, a pessoa que criara mais personagens será o Mestre, já que ele controla toda a trama que os jogadores estarão participando. Cada um dos outros jogadores criam o personagem que irão controlar. Eventualmente quando o Mestre é experiente e vai mestrar para pessoas iniciantes que só querem conhecer, o próprio Mestre cria os personagens que o os jogadores irão controlar. 

Exemplo: João decide ser o Mestre, porque ele tem uma história bem legal preparada, leu e entendeu muito bem o manual. André, Luíza e Rafael serão os jogadores, cada um deles deve criar um personagem para participar da história do João.

\section{A criação do personagem começa...}

Com a história! Primeiro é importante saber o cenário e os limites do mundo da história do Mestre, se cenário for fantasia medieval onde existem feiticeiros e dragões seu personagem pode ser de um jeito, agora se o cenário for baseado histórias futuristas onde há lutas de sabre de luz e armas laser, seu personagem será de outra maneira. Enfim, seu personagem tem que ser factível com o mundo da história do Mestre.

Agora que você compreendeu o mundo em que seu personagem viverá, agora é a vez de pensar na história do seu personagem. Onde ele nasceu? Quem eram seus pais? Possui irmãos? Qual o nome do seu personagem? Como foi a infância dele? Que habilidades ele vai ter? Com quem aprendeu as suas habilidades? Ele é sério? Bem humorado? Porque se decidiu aventurar? Como é a sua aparência (cabelos, roupas, cor dos olhos, pele)? Como ele luta? Enfim, essas são algumas questões que podem ajudar a construir seu personagem. Mas não tente pensar em tudo agora, provavelmente você terá uma ideia básica dele agora e depois das distribuições de pontos, você pode querer mudar vários aspectos(ou mudar tudo mesmo) da história para se adequar a ficha do personagem. Por falar em ficha...

Exemplo: O Mestre João diz aos seus jogadores que a aventura se passará no país de Hoveran, mundo que eles vão jogar é de fantasia medieval onde é dragões e magos não são lenda. O jogador André decide que seu personagem será um mago orfão que passou a vida inteira na academia de treinamento dos magos, Luíza diz que sua personagem é uma guerreira amazona, sua família é a tribo onde foi criada e vive nas florestas de Hoveran, Rafael decide que seu personagem é um veterando de guerra do Vietnã, que anda com uma metralhadora e um cinto de granadas. João alerta Rafael que esse tipo de personagem não faz o menor sentido no mundo que irão jogar, pois lá não existe Vietnã e nem metralhadoras ou granada. Rafael, pensa um pouco e sugere que o personagem dele é um infiltrador, que utiliza facas e furtividade para abater seus inimigos a distância.

\section{A ficha de personagem}

\section{As características}

\subsection{Atributos}

\subsection{Atributos secundários}

\subsection{Pontos de Vida, Magia e Ação}

\subsection{Vantagens e Desvantagens}

\subsection{Perícias}

\subsection{Magias}

\subsection{Pertences iniciais}

\subsection{Revisão}





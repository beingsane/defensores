\chapter{A Aventura}
\label{ch:adventure}

Nesse momento você sabe quais sãos os principais atributos,  conhece muitas vantagens, perícias e magias, informações fundamentais para criar um personagem. Mas como se joga?

Já foi explicado na introdução o papel do Mestre e dos jogadores, cada um dos jogadore deve criar e interpretar um personagem enquanto o Mestre cuida de todo universo que eles serão inseridos.

Quem incia o jogo é o Mestre, ele diz onde cada jogador está e então decide quem terá a ação. O momento de agir do jogador é chamado de turno. No turno do jogador, ele deve decidir o que seu personagem irá fazer: interagir com outros personagens, ir para algum local do jogo, procurar algo... As consequências da ação do jogador são decididas pelo Mestre. Por exemplo:

Mestre: João, seu personagem acaba de chegar no hotel.

Jogador: Ok, vou reservar um quarto.

Mestre: Você se aproxima do atendente. "Olá senhor, como posso ajuda-lo?" - Mestre interpretando o atendente.

Jogador: "Qual o preço da noite?" - O jogador interpretando o personagem.

Mestre: "Nosso quarto mais barato está \$20, senhor".

Jogador: "Ok, o preço está bom, vou querer alugar um por 2 dias."

(...)

Assim é a interação iria continuar até o Mestre decidir o turno e passaria a ação para outro jogador. Como você pode perceber, a dinâmica do jogo se parece com um teatro. O jogo inteiro poderia ser assim, com conversas e decisões, porém, para ter uma dinâmica de um jogo temos regras e fatores aleatórios. Os fatores aleatórios são decididos por rolagem de dados. Imagine que o jogador decide realizar algo de maior risco, como saltar de um prédio para o outro, atacar alguem, tentar convencer o vendedor dar um desconto... enfim, o Mestre poderia dizer simplesmente "conseguiu" ou "não conseguiu", mas perderia um fator de diversão que é a sorte decidir o evento, onde a habilidades do personagem aumentam a probabilidade de sucesso. Segue a explicação de como usar os dados e o personagem para decididir os eventos.

\section{Dados}

\section{Testes}
\begin{description}
\item[Atributos]
\item[Perícias]
\item[Disputas]
\end{description}

\section{Pertences inciais}
\subsection{Dinheiro}
\subsection{Equipamentos}

\section{Locais}

\section{Movimento}
\begin{description}
\item[Voando]
\item[Correndo]
\item[Viajando]
\item[Nadando]
\item[Escalando]
\end{description}

\section{Privações}
\begin{description}
\item[Respiração]
\item[Alimentação]
\item[Sono]
\end{description}

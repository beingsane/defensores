\chapter{A Aventura}
\label{ch:adventure}

Nesse momento você sabe quais sãos os principais atributos,  conhece muitas vantagens, perícias e magias, informações fundamentais para criar um personagem. Mas como se joga?

Já foi explicado na introdução o papel do Mestre e dos jogadores, cada um dos jogadore deve criar e interpretar um personagem enquanto o Mestre cuida de todo universo que eles serão inseridos.

Quem incia o jogo é o Mestre, ele diz onde cada jogador está e então decide quem terá a ação. O momento de agir do jogador é chamado de turno. No turno do jogador, ele deve decidir o que seu personagem irá fazer: interagir com outros personagens, ir para algum local do jogo, procurar algo... As consequências da ação do jogador são decididas pelo Mestre. Por exemplo:

Mestre: João, seu personagem acaba de chegar no hotel.

Jogador: Ok, vou reservar um quarto.

Mestre: Você se aproxima do atendente. "Olá senhor, como posso ajuda-lo?" - Mestre interpretando o atendente.

Jogador: "Qual o preço da noite?" - O jogador interpretando o personagem.

Mestre: "Nosso quarto mais barato está \$20, senhor".

Jogador: "Ok, o preço está bom, vou querer alugar um por 2 dias."

(...)

Assim é a interação iria continuar até o Mestre decidir o turno e passaria a ação para outro jogador. Como você pode perceber, a dinâmica do jogo se parece com um teatro. O jogo inteiro poderia ser assim, com conversas e decisões, porém, para ter uma dinâmica de um jogo temos regras e fatores aleatórios. Os fatores aleatórios são decididos por rolagem de dados. Imagine que o jogador decide realizar algo de maior risco, como saltar de um prédio para o outro, atacar alguem, tentar convencer o vendedor dar um desconto... enfim, o Mestre poderia dizer simplesmente "conseguiu" ou "não conseguiu", mas perderia um fator de diversão que é a sorte decidir o evento, onde a habilidades do personagem aumentam a probabilidade de sucesso. Segue a explicação de como usar os dados e o personagem para decididir os eventos.

\section{Dados}

Os tipos de dados que você usará em um jogo de RPG depende do sistema, enquanto alguns sistemas usam dados de 12 faces outros usam apenas o dado comum de 6 faces. Descrevemos a quantidade de faces de um dado ao lado da abreviação "d", por exemplo d6 é um dado comum de 6 faces, d8 é um dado de 8 faces, d10 um dado de 10 faces e etc. Para o sistema defensores você usará apenas o dado comum de 6 faces, portanto usaremos apenas anotação "d".

Dependendo do momento será necessário rolar mais de uma vez o dado, para isso, haverá um número antes da abreviação "d" indicando a quantidade de vezes. 2d significa role o dado duas vez e some seu resultado, 4d role o dado quatro vezes e some.

Enfim, pode existir também os modificadores do resultado que virá após a anotação do dano, por exemplo "2d + 3", role o dado duas vezes e acrescente 3 ao valor. "2d -2", role o dado duas vezes e subtraia 2 do resultado. "5d + H", role o dado cinco vezes e some o valor de Habilidade do personagem ao resultado. Simples não?

\section{Testes}
Testes são divertidos e importantes! Eles dão mecânica aleatória ao jogo criando ansiedade e expectativa para uma ação importante na aventura. Agora que você sabe como funciona a rolagem de dados não é difícil entender como funciona os testes.

Mas antes que você obrigue os jogadores a fazerem teste para mirar na privada ao urinar, é importante que o Mestre só peça testes para tarefas que não sejam triviais ao personagem. Por exemplo, um personagem que sabe dirigir automóveis, ele quer ir com seu carro até a padaria comprar pão, precisa de teste? Não! É uma atividade trivial para alguém com a perícia de condução. Se o personagem não soubesse dirigir, ai seria uma grande aventura chegar até a padaria.

Um teste possui um valor de "dificuldade", um personagem para passar no teste o resultado da rolagem de dado deve ser maior ou igual ao valor de "dificuldade". Por exemplo, uma teste com dificuldade 2, o jogador rola o dado e o resultado é 2, significa sucesso, pois 2 é igual ou maior que a "dificuldade" do teste. Se o resultado do dado fosse 1, o personagem iria falhar. Dificilmente apenas será o resultado, provavelmente o jogador terá modificadores de atributo, bônus ou penalidades dependendo da situação

Dependendo do momento, o jogador pode ao invés de rolar o dado pode escolher um valor. Imagine que o teste serve para ações importantes, únicas e/ou imediatas. Por exemplo, "quebrar uma porta durante uma perseguição", exige teste, enquanto ao explorar uma masmorra o personagem deseja derrubar uma porta, como ele não tem pressa, ele decide ficar ali vários minutos tentando derruba-la (ou seja, como se tivesse fazendo vários testes até conseguir), então evite esses testes repetitivos e assume que o resultado da rolagem do dado foi 6, se mesmo assim o personagem não conseguir derrubar a porta, ficará claro que é impossível.

Os principais testes são:

\begin{description}
\item[Atributos]
\item[Perícias]
\item[Disputas]
\end{description}

\section{Pertences inciais}
\subsection{Dinheiro}
\subsection{Equipamentos}

\section{Locais}

\section{Movimento}
\begin{description}
\item[Voando]
\item[Correndo]
\item[Viajando]
\item[Nadando]
\item[Escalando]
\end{description}

\section{Privações}
\begin{description}
\item[Respiração]
\item[Alimentação]
\item[Sono]
\end{description}

\chapter{Os Atributos}
\label{ch:attributes}

Os atributos são extremamente importantes pois determinam o que o personagem é capaz de fazer. Cada um deles tem sua importância e são vitais na estratégia de criação do personagem. Os atributos influenciam os atributos secundários e estão diretamente ligados ao poder de combate. 

\section{Força}

Pontos gastos em Força(F) representa que o personagem tem treinamento em combate corpo-a-corpo, ou seja, para combates de curta distância é sua principal fonte de dano, manobras e ainda determina a quantidade de kilos que ele é capaz de mover. O atributo Força não precisa ser necessariamente musculos, pode ser uma técnica de espada antiga que com seu golpe é possível mover coisas sem corta-las.

\begin{description}
\item[FA] Para combate corpo-a-corpo a Força é a principal contribuídora para para sua Força de Ataque, que será: 
\[ FA = (F \times 2) + Habilidade \]
Personagens sem pontos em força e quer lutar corpo-a-corpo, recebe -4 em sua FA. {\bf Exemplo:} Um personagem com F3 e H4, a FA final será 10(\(3 \times 2 + 4 \)). 
\item[Peso] Para cada ponto em Força, seu personagem é capaz de carregar 100kg. Personagens com F0, movem 50kg. Para mover algo no limite do peso da sua força não é necessário testes, para mover coisas além da sua força, sua velociade é reduzia a 1/4 e é necessário passar em teste de dificuldade (4 + Pontos de Força necessário). {\bf Exemplo:} Para mover algo de 700kg, é necessário passar em um teste de dificuldade 11 (\(4 + 7\)).
\item[Manobras] Há várias manobras e vantagens especiais exclusivas para personagens que lutam corpo-a-corpo, veja mais no capítulo Regras de combate(pag. \pageref{ch:combat})
\end{description}

\begin{framed}
No sistema 3D\&T o personagem era capaz de carregar muito mais peso com seus pontos em força, porém era difícil de controlar e lembrar. Com 100kg por força, o nível de poder é menor e fácil de lembrar, pois basta multiplicar o valor da força por 100kg. F3 é 300kg a capacidade, F8 é 800kg, no sistema antigo quanto era capacidade para F12 mesmo? Tem que consultar o manual, desse jeito dispensa o manual e é mais fácil de prever o que os personagens são e serão capazes.
\end{framed}

\section{Habilidade}

Habilidade(H) é a destreza e agilidade do seu personagem, esse atributo ajuda no combate em todos aspectos, é um dos atributos principais para uso de perícias, além de ser útil para testes que exigem esquiva, movimento rápido, acrobacias, saltos, etc.

\begin{description}
\item[FA] Tanto para combate corpo-a-corpo e combate a distância é somado a FA final. Portanto, em ataques corpo-a-corpo:
\[ FA = (F \times 2) + H \]
A distância:
\[ FA = (PdF \times 2) + H \]
\item[FD] Além de contribuir nas lutas corpo-a-copo, Habilidade também é somada a defesa:
\[ FD = (A \times 2) + H \]
Existem ataques que ignoram Armadura, restando apenas a sua Habilidade para te defender em sua FD.
\item[Velocidade] Habilidade contribui diretamente na movimentação do personagem durante o combate ou curtas distâncias que será
\[ Velocidade = (H \times 10m/s) \]
Personagens com H0, movem a 5m\/s. Para longas corridas ou longas jornadas, depende também da resistência do personagem, veja no capítulo "A Aventura"(pág. \pageref{ch:adventure}) 
\item[Salto] A dois movimentos clássicos para salto, salto somente na vertical e o salto  horizontal. Para o salto somente na vertical, considere que o alcance máximo em metros é: 
\[ V/2 \] 
Sendo V, Velocidade.
Para o alcance horizontal do salto horizontal em metros, considere:
\[ (AV / 2) + (V/2) \]
Sendo V, Velocidade e AV, Alcance do Salto Vertical.
Para o alcance vertical do salto horizontal em metros, considere:
\[ AV/2 \]
Sendo AV, Alcance do Salto Vertical.
\item[Perícias] Habilidade é um dos atributos chaves para várias perícias, essas perícias o personagem pode tentar usa-las mesmo que não as tenha, com um redutor de -3 em seu teste. Algumas perícias, só é permitido usá-las se houver um tutor ou treinamento nelas.
\item[Testes] Testes que exigem movimento rápido, preciso ou acrobacias, esse é o atributo! A dificuldade deve ser \( 4 + H \), sendo H o nível de habilidade necessária para realizar a tarefa sem precisar realizar testes.
\item[Manobras] A manobra mais importante que usa Habilidade como atributo é a Esquiva. Para usar Esquiva contra um ataque, o personagem deve ter Habilidade superior a do oponente. o número de vezes que um personagem pode tentar se esquivar de um ataque em uma rodada de combate é igual o número da sua Habilidade. Veja mais sobre esquiva no capítulo "Regras de Combate" (pag. \pageref{ch:combat}).
\item[Ataque Múltiplo] Um personagem com alto valor em habilidade, pode ter mais de um ataque em uma mesma rodada. Veja mais em "Regras de Combate" (pag. \pageref{ch:combat}).
\end{description}

\begin{framed}
No sistema 3D\&T Habilidade era o atributo mais importante e desbalanceado do jogo. Tão desbalanceado, que as últimas sessões que tive os jogadores investiam mais em Habilidade do que qualquer outro atributo. Pois além de ser o atributo chave de TODAS as perícias, representava também a inteligência, como também tinha o mesmo valor de combate para defesa e ataque. Visivelmente, ninguém seria bobo de não gastar vários pontos em habilidade, gerando personagens muito parecidos e com as mesmas estratégias. Por isso achei importante reduzir sua importância para equilibrar sua utilidade com outros atributos.
\end{framed}

\section{Resistência}
O atributo Resistência(R) representa condição física do seu personagem. Um alto valor em Resistência permite o personagem ter mais Pontos de Vida, Fadiga e resistir a diversas doenças, venenos, dores e efeitos mágicos. Enfim, o atributo Resistência garante que o seu personagem sobreviva a diversas situações.

\begin{description}
\item[PVs] Resistência contribui diretamente com os Pontos de Vida do Personagem, sendo:
\[PVs = R \times 5 \]
Personagens com R0, possuem 1PV.
\item[Fadiga] Cada ponto de R equivale a um ponto de Fadiga.
\item[Recuperação] Para cada dia de repouso o número de PVs recuperados é igual o seu valor de R. 
\item[Testes] Muitas situações exigem que o personagem tenha uma condição física exemplar. Resistência é útil para resistir venenos, efeitos mágicos, dores, mais tempo acordado, caminhar e correr por mais tempo, etc.
\end{description}

\begin{framed}
Para Resistência, comparando ao sistema 3D\&T, as diferênças são minimas. Por isso não tenho muito o que comentar.
\end{framed}

\section{Mente}
O atributo Mente(M), representa a raciocinio, a sabedoria, a memória, a força de vontade, etc. Um personagem com alto valor em Mente, representa as pessoas mais brilhantes do seu tempo. O atributo Mente garante Pontos de Magia, usar diversas perícias, resistir a controle mental e escapar de trapaças ou truques.

\begin{description}
\item[PMs] Mente contribui diretamente com os Pontos de Magia do Personagem, sendo:
\[PMs = M \times 5 \]
Personagens com M0, possuem 1PM.
\item[Perícias] Mente é um dos atributos chaves para várias perícias, essas perícias o personagem pode tentar usa-las mesmo que não as tenha com um redutor de -3 em seu teste. Algumas perícias, só é permitido usá-las se houver um tutor ou treinamento nelas.
\item[Magias] O poder das magias são influenciadas diretamente pelo valor do seu atributo Mente. Magos poderosos possuem um alto valor nesse atributo.
\item[Testes] Situações que exigem raciocinio, percepção, memória, força de vontade ou sabedoria, Mente é o atributo adequado para o teste.
\end{description}

\begin{framed}
Acredito que esse é o atributo mais polêmico e desejado por muitos jogadores de 3D\&T. Como essa atualização tem a intenção de trazer uma real adição ao jogo, um atributo que separe a agilidade de inteligência se faz necessário. Em 3D\&T tudo se resumia a Habilidade: combate, magia, testes e perícia. Um personagem para ser um bom mago tinha que ter Habilidade alta, mas ao mesmo tempo que ele se tornava um bom mago, ele também se tornava um acrobata épico e ótimo lutador, assim os personagens tendiam sempre ter Habilidade alta. Com essa atualização, Habilidade perde o centro da atenção, permitindo maior variadade de personagem e descentralização de compras em um único atributo.
\end{framed}

\section{Armadura}
O atributo Armadura(A) contribui diretamente com a defesa do personagem. Mesmo em situações em que seu personagem se encontra indefeso, a Armadura está lá para protege-lo. O atributo Armadura não necessariamente significa que seu personagem vista uma armadura, pode ser apenas amuleto de proteção, ou bracelete, ou até mesmo sua pele que é mais resistente que o normal.

\begin{description}
\item[FD] Armadura contribui principalmente para sua Força de Defesa, sendo:
\[ FD = (A \times 2) + H \]
Personagens com A0, não sofrem redutor em FD, apenas não tem bônus em Armadura.
\item[Passiva] Em algumas situações o personagem não pode se mover, ou seja, não pode usar sua Habilidade na somatoria do bônus da FD, entretanto, pode continuar usando o bônus da Armadura normalmente.
\item[Manobras] Algumas manobras e vantagens, dependem exclusivamente da Armadura para aumentar a proteção do personagem. 
\end{description}

\begin{framed}
Armadura sempre foi o patinho feio dos atributos principais em 3D\&T, porque não contribuia muito na proteção personagem, apesar de possuir algumas vantagens que poderia valer a pena. Nessa versão, ela se torna o principal atributo para proteção, além disso algumas vantagens que usavam habilidade para proteção, usará Armadura no lugar.
\end{framed}

\section{Poder de Fogo}

Poder de Fogo(PdF) é o atributo que representa o ataque a distância do personagem. Pode ser um revolver, bolas de fogo ou ki, metralhadora, raios, etc. O visual do seu ataque a distância só necessita da aprovação do Mestre. Resumindo, Poder de Fogo determina a força e o alcance do ataque a distância. 

\begin{description}
\item[FA] Poder de Fogo contribui principalmente para Força de Ataque a distância, mas não tem problema algum usar em combate corpo-a-corpo. Sendo:
\[ FA = (PdF \times 2) + H \]
Personagens com PdF 0, recebem -4 em sua FA a distância.
\item[Alcance]
O alcance do ataque a distância em metros será:
\[ Alcance = PdF \times 20 \]
Mesmo que o PdF de um personagem não seja suficiente para atingir a distância desejada, ele pode realizar o ataque com redutor na FA. O redutor é de -2 na FA, para cada pontos de diferença para o PdF necessário. Exemplo: Um personagem com PdF2 a quer atacar um personagem a 80 metros, porém para atingir a 80 metros sem redutor é necessário PdF 4, logo diferença é 2 pontos, portanto o redutor na FA será de -4.
\item[Testes]
Testes que necessitem verificar a pontaria do personagem, Poder Fogo é o atributo correto a ser testado.
\item[Manobras]
Existem manobras e vantagens que usam exclusivamente Poder de Fogo que ajudam combater melhor a uma distância segura.
\end{description}

\begin{framed}
Também não houve muitas mudanças, o alcance do ataque a distância foi reduzido e adicionado uma formula simples de calcular sem precisar de muita consultas. Agora assim como Força, Poder de Fogo é mais importante que Habilidade para FA.
\end{framed}

\section{Atributos Secundários}

Força, Habilidade, Resistência, Mente, Armadura e Poder de Fogo sãos o principais atributos para construção de um personagem, além deles, existem os Atributos Secundários, que são diretamente influenciados pelos atributos principais e podem ser aprimorados através de vantagens.

\subsection{Pontos de Vida}

Pontos de Vida(PV) representam a vitalidade do personagem, ou seja, o quanto seu personagem resistirá a danos até ficar incapacitado de lutar. Os pontos de vida limite do personagem são calculados a partir da Resistência (R) do personagem e pode ter bônus(B) adiquirido de alguma vantagem, item mágico e etc. O cálculo é basicamente:

\[ PVs = (R + B) \times 5 \]

Personagens com R0 possuem 1Pv. Enquanto o personagem possuir pontos de vida ele pode agir normalmente. Os PVs quando perdidos, podem ser recuperados através itens de cura, magias, repouso ou tratamento médico. Enquanto a recuperação por magia o efeito é instantâneo, por meios naturais é mais demorado, precisando de dias para que o personagem fique totalmente recuperado. Existem dois tipos:

\begin{description}
\item[Repouso] Descanso de 8 horas seguidas. Quando o personagem descansa dessa maneira ele recupera \( R/2 \) PVs. No minímo de 1PV.
\item[Repouso completo] Descanso por 24 horas seguidas. Quando o personagem descansa dessa maneira ele recupera \( R \times 2 \) PVs.
\end{description}

Um personagem ao recuperar seus PVs nunca deve ultrapassar seu limite, o limite de PVs só deve ser expandido com vantagens ou aumentando a resistência. {\bf Exemplo } Heryum é um guerreiro que possui R2 e 10PVs, decidiu viajar e durante a sua viagem foi atacado por ladrões, com muito sacrificio os ladrões foram derrotados e Heryum agora tem apenas 2PVs. Anoitece, ele procura um local seguro e dorme, nesse repouso ele recupera 1PV (\( R / 2 \)). Após acordar ele procura uma vila próxima para se recuperar melhor, por sorte ele encontra e vai direto para uma estalagem ficar em repouso absoluto. Após um dia de repouso absoluto ele recupera 4PVs(\( R \times 2 \)).

Quando um personagem atinge a 0 Pvs ou menos, ele ficará incapacitado. É um dos momentos mais dramáticos para um personagem é quando ele está chegando perto da morte, por isso há algumas regras para esse momento. Um personagem que leva um dano que reduz seus PVs a zero ou menos, deve antes checar a sua resistência, se esse o dano absoluto que ultrapassou os Opvs for maior que a resistência ele permanecerá com esses PVs negativos, se não mantém-se com 0PVs. {\bf Exemplo } Um personagem com R2 e atualmente com 2PVs, recebeu 4 pontos de dano, o personagem ficaria com -2PVs (\( 2 -4 \) ). Porém, o valor absoluto de seus PVs negativos é 2, que não é maior que o valor da Resistência do personagem, portanto ao invés de -2PVs, o personagem ficará com 0PV. Se no mesmo caso o dano fosse 5, o personagem ficaria com -3PVs, esse valor em absoluto é maior que a resistência do personagem, portanto ele se manteria com seus -3PVs.

Após o personagem ficar incapacitado, ele fica indefeso, e qualquer dano que o personagem levar agora subtrai normalmente dos seus PVs atuais. Para cada nível de dano abaixo dos 0PVs existe uma condição que o personagem se encontra.


\begin{description}
\item[Incapacitado] O personagem está nessa condição quando ele possui 0pv. Ele ainda pode falar, se rastejar e fazer ações extramente simples. Qualquer atitude acima disso ele ficará automaticamente com -1PV, portanto no próximo estágio Inconsciente.
\item[Inconsciente] Um personagem inconsciente não tem muito o que fazer. Se ele sofrer algum dano, e o valor de seus PVs em absoluto for mario que o valor da sua Resistência, ele mudará a sua condição para Ferido.
\item[Ferido] Nessa condição o personagem está morrendo, mas lentamente. A cada uma hora os PVs do personagem serão reduzidos em 1, até os PVs do personagem em valor absoluto ser maior que o valor da \( R \times 2 \). Quanto atinge esse nível, o personagem está Morrendo! A perícia Medicina pode impedir essa redução de PVs.
\item[Morrendo] Nessa condição a vida do personagem se esvai muito mais depressa. Ele perde 1PV a cada minuto do jogo, até seus PVs em absoluto ser igual ao seus PVs. Quando isso acontece, seu personagem está morto. A perícia Medicina pode impedir essa redução de PVs.
\item[Morto] Como o próprio nome diz, o seu personagem morreu. Ele pode voltar como um Fantasma, morto-vivo, seus companheiros podem te ressucitar, ou você pode criar um novo personagem para substituir esse que morreu.
\end{description}

\begin{framed}
Sim! Pontos de Vida negativo! Antigamente para determinar se o personagem estava morto era pela sorte no dado, muitas vezes um golpe fulminante que deveria matar o personagem o mantinha vivo e um golpe simples poderia matar. Mudei essa regra para que a resistência e os PVs do personagem determinem sua condição. Não importa sua condição, a perícia medicina pode manter o personagem vivo.
Além da adição dos PVs negativos, também há a dificuldade em recuperar PVs. No sistema antigo, bastava uma noite de sono para recuperar todos os PVs. Observando um pouco melhor vários animes, mangas, HQs, séries, filmes, e etc, é mais comum um personagem levar vários dias para se recuperar totalmente do que se recuperar espantosamente em uma única noite. A Resistência do personagem é fundamental para uma recuperação mais rápida.  
\end{framed}

\subsection{Pontos de Magia}

Pontos de Magia(PM) representam a energia mágica, espiritual, física, ou ki do personagem. Para ativar habilidades especiais ou lançar magias é necessário gastar PMs. Quando um personagem chega a 0PM, ele é fica incapaz de usar sua habilidades, não é possível ter PMs negativos. Assim como PVs, o personagem pode recuperar os PM's gastos, porém somente com itens e descanso.

\begin{description}
\item[Repouso] Descanso de 8 horas seguidas. Quando o personagem descansa dessa maneira ele recupera \( M/2 \) PMs. No minímo de 1PM.
\item[Repouso completo] Descanso por 24 horas seguidas. Quando o personagem descansa dessa maneira ele recupera \( M \times 2 \) PMs.
\end{description}


\subsection{Pontos de Ação}

Pontos de Ação (PA) representam a sorte, a força de vontade sobrenatural ou qualquer coisa que permita seu personagem fazer do impossível, possível. Os PAs não são influenciados por nenhum atributo, eles funcionam mais como sorte ou força de vontade ao máximo. Porém esses pontos são extremamente escassos e raros de se conseguir, todo personagem pode possuir no máximo 3 PAs. Na criação do personagem é normal atribuir aleatoriamente os PAs para os jogadores, jogue um dado, divida o resultado por 2 arredondado para baixo e no minímo de 1, esse é o valor de PA para o jogador. Assim, um jogador pode começar com 1PA ou no máximo 3PAs. Para saber como um jogador consegue mais PAs, veja no capítulo 'Recompensas' (pag. \pageref{ch:rewards}). Alguns exemplos que o personagem pode fazer com PAs:

\begin{description}
\item[Testes] Se o Mestre te solicitar um teste, você gastar 1PA para ter um bônus + 3 nesse teste. 
\item[Falhas] Se caso falhar em um teste, você pode gastar 1PA para ter uma nova chance.
\item[Critíco] Por 1PA você pode garantir um ataque ou defesa crítica.
\item[Recuperação] Por 1PA você pode recuperar \(RX2\) Pvs ou \(MX2\) PMs.
\end{description}

Esses são alguns exemplos bem abrangentes, o Mestre e os Jogadores podem ter mais idéias de como utilizar os PA, mas sempre, com a aprovação do Mestre.

\begin{framed}
No 3D\&T clássico, no lugar de PAs, os jogadores tinham que gastar Pontos de Experiência para conseguir realizar essas tarefas extraordinárias. Enquanto uns acumulavam experiência para evoluir, outros gastavam para ajudar o grupo ou a si mesmo. Tendo momentos totalmente desquilibrados onde um jogador com 9 PEs poderia ter 9 criticos seguidos e ainda recuperar seus PVs. PAs com limite de 3, vem para impedir esse desiquilibrio e ainda permitir essa vantagem aos jogadores sem sacrificar a evolução deles. 
\end{framed}

\subsection{Força de Ataque}

Força de Ataque é o principal parâmetro para medir o ataque de alguma coisa contra a defesa de outra coisa. Seu cálculo é simples:

\[ FA = (F \times 2) + H \]

Ou:

\[ FA = (PdF \times 2) + H \]

Porém pode variar devido algum bônus de alguma vantagem ou por ser alguma magia. Depois do cálculo da FA, rola-se um dado(1d) e soma o valor a FA. Então:

\[ FAfinal = FA + 1d \]

Se o resultado da rolagem do dado for um 6, então o ataque é um critíco e a FA final é calculada diferente:

\[ FAfinal =  FA + (FA/2) + 6 \]

Ou seja, um critico aumenta em 50\% a FA, não deve incluir o  resultado do dado, mas pode incluir os bônus de vantagens, itens, magias, etc. {\bf Exemplo: } Um personagem com F3 e H2, sua FA será 8 (\(F \times 2) + H(\)), ele rola um dado e o resultado foi 2, então sua FA final será 10 (\( FA + 1d \)). Para o mesmo caso, se ele tivesse tirado um 6, então foi um critico, assim sua FA final será 18 (\(FA + (FA/2) + 6\)).

\begin{framed}
Além da já comentada dobrada Força e Poder de Fogo na FA, a maneira de calcular o critíco mudou. No antigo 3D\&T apenas o atributo chave era dobrado, em Defensores todos os bônus de atributos e vantagens ajudam mas o aumento é em escala menor (50\%).
\end{framed}

\subsection{Força de Defesa}

Força de Defesa é o principal parâmetro para medir a defesa de alguma coisa contra o ataque de outra coisa. Seu cálculo é simples (Copy paste?):

\[ FD = (A \times 2) + H \]

O resto funciona praticamente igual a FA, rola-se um dado para ter a FD final, se houver um 6, calcula-se o critico da mesma maneira.

\subsection{Fadiga}

\subsection{Velocidade}

\subsection{Alcance de Poder de Fogo}

\subsection{Tipo de Dano Especializado}

\subsection{Experiência}

\chapter{Regras de combate}
\label{ch:combat}

Uma jogo de RPG não precisa haver combate nenhum, simplesmente pode ser uma história onde há resolução de enigmas, investigação, interação social ou desvendar mistérios, não sobrando espaço para um combate. Se o Mestre preferir, pode haver combate com suas regras próprias para decidir rapidamente quem ganhou. Mas, tem aqueles que querem jogar RPG e ter um grande momento de combate, onde cada decisão no combate é importante e o resultado dos dados ajudam a deixar tudo imprevisível e emocionante. Sim, se você estava procurando regras para combate, está aqui tudo que você precisa para ter um combate em Defensores!

\section{O começo, o meio e o fim}

O combate começa quando há um encontro onde a batalha é inevitavel. "Surgem orcs da colina gritando e correndo na direção de vocês", "Eu decido atacar o guarda da cidade", "Vou quebrar a garrafa na cabeça do fulano para começar uma briga no bar"... são exemplos momentos que o confronto inicia. A regra padrão é no início de cada confronto joga-se as iniciativas de cada personagem(jogadores e não jogadores), os valores mais altos significam quem poderá agir primeiro.

\[ Iniciativa = 1d + H + Bônus \]

Anote esses valores de iniciativas, pois eles podem mudar durante o combate. Ordene os personagens que irão agir de acordo com a iniciativa, caso haja iniciativas iguais considere que os persoangens agem ao mesmo tempo no efeito de jogo. Se sua aventura terá muitos combates você pode agilizar fixando o valor da iniciativa dos personagem por sessão de jogo.

Após os personagens ordenados, cada um deles poder realizar alguma ação na sua vez, essa "vez", chamamos de "turno". Quando o último personagem da ordem de iniciativa agir, diz que acabou a "rodada". Portanto, uma rodada termina quando todos os personagens realizaram seu turno. Mas o que pode se fazer em um turno de combate?

Em um turno de combate o personagem pode realizar 2 tipos de ações:

\begin{description}
\item[Ação] Ação é um ataque, usar alguma vantagem ou perícia, lançar uma magia, usar um item, trocar de armas, realizar alguma manobra de combate ou qualquer outra coisa que o jogador imaginar.
\item[Movimento] O movimento é o deslocamento no combate, se aproximar ou se afastar de um inimigo, trocar de armas, usar um item, levanter-se de uma queda. Sobre os variados tipos de deslocamento e sua velocidade já foi descrito no capitulo Aventuras.
\end{description}

Um personagem em seu turno não precisa realizar a sua Ação e o seu Movimento, se quiser, apenas pode realizar um movimento ou uma ação. Ainda, se o personagem quiser, ele pode usar a sua Ação para realizar um Movimento adicional mas nunca trocar um Movimento para realizar uma ação. Resumindo, em um turno o personagem pode realizar uma "Ação", um "Movimento", "Ação e Movimento" ou "Movimento e Movimento". Lembrando que após uma rodada de "Movimento e Movimento", se o personagem persistir n ação de "Movimento" a velocidade será limitada pelo atributo Resistência.

Falar durante a batalha não consome ação, é livre, simplesmente para promover a interpretação.

Por motivos óbvios, um personagem incapacitado, dormindo, morto ou paralisado não poderá agir em seu turno.

O combate quando todos oponentes ou jogadores foram derrotados, oponentes ou jogadores conseguiram fugir, ou simplesmente quando todos decidiram parar de lutar por algum motivo.

\section{O ataque}

O ataque é bastante simples, o personagem escolhe se usará o atributo Poder de Fogo ou Força para atacar o oponente, mesmo que esse atributo tenha um valor 0. O personagem somente poderá usar a Força se estiver pelo menos a 3 metros do oponente. Se estiver muito distante, ele pode usar uma ação de Movimento para se aproximar. As vezes o alvo do ataque está tão distante que mesmo usando a movimentação para se aproximar ainda não é suficiente, lhe restando apenas Poder de Fogo para atacar. 

Após escolhido o atributo de ataque, o personagem deve decidir se realizará alguma manobra de ataque ou usará alguma vantagem. Ao menos que a manobra ou vantagem especifique outro efeito, a batalha prossegue com o cálculo da FA final, lembre-se que existem itens, efeitos mágicos e vantagens que aumentam a FA. Imediatamente o alvo deve realizar o cálculo da FD. Os pontos de vida perdido pelo alvo será:

\[ PvsPerdidos= FA - FD \]

Onde o mínimo de PVs perdidos será 0, ou seja, se o alvo obter um valor de FD maior ou igual a FA do atacante, ele não sofrerá dano algum. Exemplos: FA 15, FD 12, causa 3 pontos de dano no alvo. FA 15, FD 15, não causa pontos de dano no alvo. FA 15, FD 17, não causa pontos de danos no alvo.

\subsection{O ataque crítico}

O ataque crítico acontece quando um personagem consegue uma valor 6 na rolagem de dado durante o cálculo da FA final. O crítico faz com que a FA seja 50\% maior. Não use o valor da rolagem de dado para calcular o crítico.

\subsection{Trapalhadas}

Dependendo da situação o personagem pode ter problemas para atacar. As vezes ele está enfraquecido por algum veneno ou efeito mágico, ou está lutando em cima de um terreno perigoso ou tendo que lutar enquanto existe um refém. Existe bastante situações que pode prejudicar um personagem enquanto luta, o Mestre com seu bom senso, deve decidir o valor apropriado de redutor na FA de acordo com a situação. Uma situação comum é quando por algum motivo o personagem não pode mover-se mas ainda pode agir usando as mãos, nesse caso não use o valor da H no cálculo da FA.

\section{A defesa}

Ao sofrer um ataque, o personagem deve imediamente realizar o cálculo da sua FD para reduzir ou eliminar o dano.

\subsection{A defesa crítica}

O defesa crítico acontece quando um personagem consegue uma valor 6 na rolagem de dado durante o cálculo da FD final. O crítico faz com que a FD seja 50\% maior. Não use o valor da rolagem de dado para calcular o crítico.

\subsection{Trapalhadas}

Da mesma maneira que foi descrita no ataque, muitas situações podem fazer o que o personagem tenha um redutor na sua FD. Porém, existem algumas situações que são bastante comuns:

\begin{description}
\item[Supreso] O alvo foi surpreendido por um ataque. Nesse caso, não use o valor de H no cálculo da FD.
\item[Toque] Alguns ataques não precisa penetrar a defesa do alvo, um simples toque já é o suficiente. Nesse caso não use o valor da A no cálculo da FD.
\item[Sem movimentação] Por algum motivo o alvo não pode mover-se, está com suas pernas presas ou defendendo alguma posição, nesse caso, não use o valor de H no cálculo da FD.
\item[Imobilizado] O personagem está paralisado, dormindo ou inconsciente. Não use o valor da H e nem a rolagem de dado para o cálculo da FD.
\end{description}

\section{Manobras}

\subsection{Ataque múltiplos}

No momento do ataque o personagem pode escolher realizar vários ataques. Para cada ataque extra, o personagem sofre um redutor cumulativo de -2 no valor de sua H para calcular a FA. O personagem não pode realizar um ataque extra se o redutor cumulativo for maior que o valor da H do personagem. Por exemplo, um personagem com H 3, pode realizar 2 ataques, um com H3 e outro com H1. Enquanto um personagem com H4, pode realizar 3 ataques, um com H4, outro com H2 e enfim um com H0. O personagem pode escolher outros alvos durante os ataques que estejam dentro do seu alcance.

Quando um personagem consegue bloquear todo o dano ou esquivar de um ataque múltiplo, todos os ataques posteriores são interrompidos, ou seja, personagem defensor conseguiu sair da linha dos ataques seguidos. Situações que o personagem defensor não pode usar H para se defender, também não conseguiŕa evitar os ataques múltiplos. 

\subsection{Ataque de misericórdia}

Quando um personagem está dormindo, imobilizado, paralisado ou de alguma outra forma indefeso, o ataque pode escolher disferir um ataque que eliminara o alvo com um único golpe. Para isso, é necessário que ele passe uma rodada completa se concentrando, no seu próximo turno, ele desfere seu ataque mortal, em efeitos de jogo esse ataque é um crítico automático.

\subsection{Ataque surpresa}

Atacar um alvo de maneira que fique surpreendido não é uma tarefa fácil, o alvo não deve ter notado a presença do atacante, deve ser realizado antes do combate iniciar e só funciona uma vez por combate. O atacante deve realizar um teste da perícia Furtividade contra o atributo Mente(ouvir ou ver) do oponente. Dependendo da situação o alvo ganha bônus ou sofre redutor para perceber o seu carrasco.

\begin{description}
\item[Alvo dormindo] Nessa situação o alvo sofre um redutor -2 em seu teste de Mente para perceber a presença do atacante.
\item[Alvo distraído] Situação padrão, o alvo está em sua rotina normal e não está esperando um ataque. O alvo não ganha bônus e nem redutor.
\item[Alvo em guarda] O alvo precisa proteger algum lugar contra invasores, ganha +2 em seu teste de Mente.
\item[Alvo em alerta] O alvo sabe que existe um intruso no local e o está procurando, ganha + 4 em seu teste de Mente.
\end{description}

\subsection{Agarrar}

Agarrar o oponente é uma manobra exclusiva do atributo Força. Para iniciar essa manobra, o personagem atacante deve realizar um ataque simples(esse ataque não causa dano) com -2 em sua FA e o alvo deve estar no alcance de combate corpo-a-corpo. O personagem conseguirá agarrar alvo se vencer a FD do alvo e um teste de disputa de Força contra o alvo. Na situação de agarrado, ambos personagens não podem se movimentar, não podem usar PdF para atacar e não podem usar Habilidade na FD e na FA. Se o personagem quiser se livrar dessa situação, ele pode realizar uma ação para realizar um novo teste de disputa de Força. Se algum dos personagens quiser se movimentar mesmo agarrado, deve vencer uma disputa de Força e se movimentara em 1/4 da sua velocidade normal. Qualquer valor de dano causado que seja maior que valor da Resistência em um personagem que está agarrando, livra a vítima do agarrão. 

\subsection{Ataque especial}

Todo personagem pode ter uma manobra secreta que aumenta instantâneamente sua FA. Ao gastar 1PM o personagem aumenta sua FA em +2.

\subsection{Ataque concentrado}

Se o inimigo for muito forte, o defensor as vezes precisa concentrar seu ki, melhorar a pontaria ou acumular forças. Enfim, o personagem pode gastar uma rodada completa para ganhar +2 em sua FA. Esses bônus são cumulativos, o número máximo de rodadas de concentração que o personagem consegue acumular é igual o valor da sua Resistência. Exemplo: Um personagem com R2, consegue no máximo em 2 rodadas +4 de bônus. Qualquer dano que o personagem receba, o bônus será perdido.

\subsection{Choque de energia}

Se o personagem ainda não realizou nenhuma ação na rodada, ele pode gastar seu turno para usar seu PdF para revidar/proteger de um ataque de Poder de Fogo. A FA vencedora atacará o personagem da FA defensora, ele deve se proteger normalmente.

\subsection{Cercar o oponente}

Muitas vezes para derrotar um personagem muito poderoso é necessário aproveitar a vantagem numérica para atrapalhar seus movimentos de defesa. Um personagem consegue enfrentar no máximo \( H \times 2 \) oponentes ao mesmo tempo sem sofrer um redutor -2 na FD. Para um personagem de tamanho  normal, é necessário pelo menos 8 oponentes do mesmo tamanho a distância corpo-a-corpo para cerca-lo totalmente e impedir seus movimentos. Quando há esse cerco total, o personagem não pode usar sua Habilidade na FD.

\subsection{Cobertura}

Muitas vezes para se proteger de ataques a distância é necessário usar coberturas para se proteger melhor. Uma cobertura parcial oferece ao jogador +2 em sua FD. Uma cobertura completa impede o ataque a distância, pois o atacante não consegue ver onde está o seu personagem. O atacante ainda pode tentar um tiro de sorte, role 1d, se cair um resultado 5 ou 6 o atacante conseguiu acertar. Existe a ocasião de o atacante conseguir enxergar o personagem em cobertura completa através de sentido especiais, algum reflexo, ou dica de alguém que indicar a posição do alvo. Um personagem atacado mesmo através de sua cobertura completa recebe +4 em sua FD. Dependendo da resitência da cobertura completa o bônus pode ser menor ou maior.

\subsection{Defender personagem}

Quem nunca viu uma cena heróica que um personagem entra na frente do outro para receber um ataque? Protege um mago enquanto ele prepara uma magia poderosa? Protege um amigo enquanto ele concentra todo seu poder?

Para pular na frente de um ataque o personagem não pode ter agido na rodada, dessa maneira ele gasta seu turno para defender o outro personagem. O personagem que será protegido deve estar dentro do seu alcance de movimentação. Ao se jogar na frente, o personagem defende o ataque sem H em sua FD.

Um personagem em seu turno normal, ele pode decidir proteger alguém, nessa posição protetora ele não pode usar sua movimentação. Ele ainda pode atacar oponente, desde que esteja dentro do seu alcance. Porém todo ataque recebido não poderá usar H em sua FD. 

\subsection{Derrubar}

Derrubar o oponente faz com que ele fique com -2 em sua FD até que ele consiga se levantar gastando uma ação de Movimento e é possível atravessar o caminho que ele protegia. Para desequilibrar o oponente e causar sua queda, é necesśario que o personagem realize um ataque simples com -2 em sua FA. Se vencer, o ataque não causa dano e derruba o oponente. 

\subsection{Desarmar}

Remover uma arma ou item que o personagem esteja usando em suas mãos pode apresentar uma grande vantagem. Para desarmar um oponente, é necessário que o personagem realize um ataque simples com -4 em sua FA. Se vencer, o ataque não causa dano e remove o item da mão do alvo. É necessário uma ação de movimento para resgatar o item caído.

\subsection{Empurrar}

Lançar um oponente longe pode ser usado como uma estratégia para vencer alguns tipos de inimigos. Para empurar um inimigo, é necessário que o persoanagem realizar um ataque com Força simples com -2 em sua FA. Se vencer, o ataque não causa dano, porém dano que seria causado multiplique por 3 metros, essa será a distância que o alvo será empurrado. 

\subsection{Esquiva}

Se o personagem preferir, antes de calcular sua FD, ele pode tentar um teste esquiva para evitar totalmente o ataque. Para isso o personagem deve ter um valor de H atual maior do que a do oponente. Esse teste funciona um pouco diferente do que o convencional, deve-se subtrair o valor de H do atacante, o defensor rola 1d, se o valor for menor ou igual do que o resultado da subtração, ele consegue se esquivar. Exemplo, o atacante possui H2, o defensor possui H4, para o defensor conseguir se esquivar, ele precisa de um resultado 1 ou 2 na rolagem do dado. O número de esquivas que o personagem pode realizar em uma rodada é igual o seu valor de H. Situações que o personagem não pode usar H na FD, significa também que não pode usar a manobra esquiva.

\subsection{Imobilizar}

Após agarrar um oponente, o personagem pode querer imobilizar o alvo, para isso, requer uma ação de rodada completa. Para imobilizar o atacante deve vencer uma difícil disputa de Força, com -2 em seu teste. Se vencer, seu oponente está imobilizado. O personagem que o mantém imbolizado não pode ataca-lo. A cada turno do personagem imobilizado, ele pode tentar se soltar obrigado novamente a disputa de Força. Da mesma maneira que agarrar, qualquer dano no atacante que supere o valor da sua Resistência liberta a vítima.

\subsection{Preparar ação}

Muitas vezes o personagem não quer agir no seu turno afim de uma estratégia melhor, ele pode aguardar e agir depois. Ele não pode guardar sua ação para próxima rodada, ele deve agir até o final da rodada, se não, perde o turno.

\subsection{Posição defensiva}

Um personagem pode se concentrar na defesa ao invés usar sua ação para realizar um ataque, nessa posição o personagem ganha +2 em sua FD.

\subsection{Fugir}

Nem sempre vencer o oponente é possível, sendo assim é necessário fugir e deixar a batalha para outro dia. O personagem deve usar seu turno para escapar. O combate só encerra quando o personagem consegue despistar seus oponentes. Para isso é necessário que ele consiga fazer que seus inimigos o percam de vista, isso é possível sendo mais veloz, entrando em becos cheio de birfucações, indo para salas cheia de portas e caminhos alternativos. Enfim, use sua imaginação para deixar a fuga emocionante. Lembre-se que enquanto o persoangem estiver fugindo ele está na verdade realizando duas ação de Movimento.

\subsection{Sacrifício Heróico}

Alguns personagem possui alguma técnica secreta que suga a própria vida em troca da vitória da batalha ou o personagem amarra explosivos em volta do corpo e explode junto com o oponente. Enfim, o personagem pode realizar manobra de ataque normal, sem gastar PM's e com acerto crítico automático. O alvo é considerado imobilizado contra esse ataque. Após esse ataque, o personagem que o desferiu morre.

\subsection{Sufocar}

Após imobilizar o personagem, se o atacante quiser ele pode sufocar a sua vítima até a inconsciência ou a morte. Para sufocar o alvo o atacante deve vencer uma difícil disputa de Força, com -2 em seu teste. Cada vez que ganhar, a vítima perde 1 ponto de fadiga até atingir 0 e ficar inconsciente. Se o personagem que quiser matar o alvo afixiado, funciona como a privação de respiração(veja no capítulo Aventuras). Um personagem ficará inconsciente por 10 minutos, ou até ser acordado ou levar qualquer ponto de dano, quando acorda recupera automaticamente 1 ponto fadiga. 


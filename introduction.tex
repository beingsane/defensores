\chapter{Introdução}

O sistema Defensores tem sua base principal no sistema de RPG chamado 3D\&T. Ele é voltado para iniciantes ou para experientes que queiram um jogo de início rápido. Nesse capítulo tentarei explicar o que é RPG, o que é 3D\&T e finalmente o que é Defensores. 

\section{O que é RPG?}

É difícil explicar o que é RPG, ele pode ser considerado um jogo, mas um jogo normalmente tem competição. Sendo uma competição, alguns jogadores deveriam ganhar e outros perderem, porém em uma sessão de RPG - nem é chamado de partida - o objetivo é todos jogadores se divertirem, ou seja, não há necessariamente ganhadores e perdedores. Os jogos de RPG possuem regras e mecânicas de jogo, mas o seu real objetivo é interpretar personagens para construir histórias. 

A sigla RPG vem Role-playing game, traduzindo, Jogo de Interpretação de Personagens. O primeiro de todos foi o D\&D (Dungeons \& Dragons). O D\%D  teve como principal insipiração os jogos de estratégia, porém expandiu com a possiblidade interpretar papéis individuais e não se limitar aos tabuleiros. O tempo foi passando, e hoje, existem vários livros de regras de RPG, desde livros com regras genéricas como esse, como específicos a um tipo de cenário ou história.

Você deve ter ouvido falar, ou até mesmo jogado, algum RPG eletrônico. Jogos como Final Fantasy, World of Warcraft, Ragnarock, Mass Effect, Dragon Age, Star Ocean, Dragon Quest, Xenogears, Cabal... enfim, existem muitos que possui o nome RPG em seu genêro. Antigamente a principal característica desses jogos era o estilo de batalha por turno, uma certa customização do personagem e muita história. Hoje em dia evoluíram bastante, permitindo que o jogador mude totalmente o rumo da história através de suas conquistas, falhas e decisões.

Ok até aqui? Que tal agora entender como ele funciona? Primeiro, temos que ter um jogador especial chamado Mestre, ou DM (Dungeon Master), ele é responsável de aplicar as regras, preparar, conduzir a história, criar e controlar outros persoangens que fazem parte dela. Cada um dos outros jogadores irão construir e conduzir um personagem. O jogo começa com o Mestre narrando o início e a partir dessa narração cada jogador tem a sua vez de decidir o que seu personagem irá fazer. O jogo se prolonga com os jogadores dizendo suas ações e o Mestre dizendo as consequências, as vezes essa consequência é determinada pela aleatoriadade do resultado da rolagem de dados. O jogo termina quando o Mestre decidir, normalmente quando os jogadores alcançam ou falham na conquista de algum objetivo determinado pelo Mestre. Do que precisa para jogar? Basicamente um manual de regras como esse, lápis, papel e dados(no caso de Defensores só necessita de dados comuns de 6 faces).

Se você ainda não conseguiu entender, não se preocupe, nas próximas páginas terá um exemplo bem didático de como o jogo funciona.

\section{O que é 3D\&T?}

Os próximo parágrafo é Copy/Paste do Manual 3D\&T Alpha Revisado criado por Marcelo Cassaro, disponibilizado para download e vendido em versão impressa pela editora Jambô. 

\textsl{ 3D\&T nasceu em 1994. Começou com poucas páginas, muito simples, como tantos outros nos Esta-
dos Unidos — onde RPGs em forma de fanzine existem aos milhares. Cinco atributos básicos, um punhado de vantagens e desvantagens. E um tema da moda: em plena febre dos Cavaleiros do Zodíaco, o jogo era uma
sátira aos heróis japoneses e seus golpes mirabolantes, ataques acompanhados de gritos, e estranhos códigos de honra. Daí o nome, Defensores de Tóquio.
}

O famoso 3D\&T começou como Defensores de Tóquio(D\&T), AD\&T, 3D\&T, 3D\&T Revisado e Ampliado, 3D\&T Turbinado, 4D\&T e o atual 3D\&T Alpha. Durante todos esses lançamentos a essência sempre foi a mesma, sendo que apenas 2 manuais tiveram grandes mudanças e novidades: o manual Turbinado e o 4D\&T. 

O manual 4D\&T particulamente, foi que menos gostei(acho que muitos jogavam/jogam 3D\&T também) pois tenta se adaptar ao sistema D20 e perde a essência do 3D\&T. O último lançamento, o manual Alpha, foi muito bom por voltar a sua essência e ao mesmo tempo um pouco decepcionante por ter demorado tanto e não trazer grandes novidades.

Apesar de eu ter ficado um pouco decepcionado com o Alpha, Cassaro, alerta logo no início do manual que realmente não tem grandes novidades. Ele diz que só houve pequenas melhorias, mas ainda acho discutível se só teve melhorias... Casso ainda tem um tempo escasso e não gosta muito do jogo, porém, graças aos fiéis jogadores, ele continuará publicando conteúdos para incrementar o sistema. Além disso, ele afirma que o sistema agora é aberto, assim qualquer pessoa pode usar o sistema para criar suas publicações.

Enfim, esse é o cenário atual do sistema, sua última versão é a Alpha, com leve contribuições do autor e ainda fortemente suportado e jogado pela galera.

\section{O que é Defensores?}

Defensores é um projeto idealizado por Ulisses Almeida, ou seja eu, em trazer ao sistema 3D\&T uma real atualização no sistema sem perder sua essência. O objetivo dessa atualização é permitir maior flexbilidade, liberdade, opções e equlíbrio ao construir um personagem, além disso, desejo trazer regras que facilite a memorização e a improvisação. Além disso não desejo lançar a versão final só com as minhas opiniões e dos meus amigos, inspirado pela filosofia do autores do RPG Pathfinder, desejo que essa atualização tenha feedback de pessoas do Brasil inteiro que gostem de 3D\&T. 

Para construir esse sistema, não tive apenas 3D\&T como base de insipiração. Jogo RPG faz 11 anos, joguei GURPS, Daemon, AD\&D, um pouco de Vampiro e os meus preferidos: 3D\&D e 3D\&T. Não existe um jogo de RPG melhor, todos eles tem suas vantagens e desvantagens, por causa disso pode ser que você ame um e odeie outro. 

A principal vantagem do 3D\&T em relação aos demais é a sua simplicidade e facilidade de aprendizado, ou seja, em poucos minutos jogadores experientes criam seus personagens, história e jogam, enquanto os jogadores iniciantes conseguem acomapanha-los sem muita dificuldades. A desvantagem do 3D\&T é quando ocorre momentos um pouco mais complexos ou realista fica difícil de adaptar, além disso há desequilibrio de vantagens e atributos, e ainda quando os personagens evoluem demais as batalhas ficam muito demoradas e repetitivas.

Defensores mantém a simplicidade, porém traz uma "encorpada" ao sistema. Com um modificações e novidades nas regras, vantagens, desvantagens, perícias, manobras e ações. Enfim, Defensores realmente se parece com uma evolução do nosso querido 3D\&T. Segue abaixo, uma tabela de comparação básica entre o sistema 3D\&T e os Defensores, para você ter leve idéia das mudanças e das coisas legais que você encontrará.

********* TABELAAAA ****************

O objetivo do Defensores é ser uma real atualização, para deixar o jogo mais divertido e interessante. Mas como todo jogo de RPG, possui suas vantagens e desvantagens, portanto pode acontecer de você odiar o Defensores. Porém não tem problema, não ficarei chateado, com os ótimos sistemas que existem por ai, com certeza você irá encontrar um que te agrade. 

\subsection{Então, eu posso ajudar?}

Claro que você pode! Essa versão do manual além de proporcionar um jogo completo, também tem o intuito de recolher feedback dos jogadores e levantar discussões sobre a mecânica do jogo. Esse é o primeiro livro que estou escrevendo e não sou formado em letras, portanto pode haver muitos erros genéricos de português e digitação. Portanto, você pode contribuir com esse projeto com elogios, críticas e ainda a escrita do manual.

Para contribuir é muito simples, você pode entrar no blog(defensores.blogger.com.br), página do facebook ou ainda mandar mensagens para o twitter do Defensores. Além disso você pode se cadastrar no github, e fazer comentários bem específicos e se souber mexer, pode realizar alterações, enviar push request para mim com comentários sobre a sua alteração.

O manual está sendo escrito usando Latex (lê-se latec). Latex é uma ferramenta bastante utilizada para escrever artigos científicos, sua vantagem é que o texto fica bastante estruturado e independente de formatação. Escolhi essa ferramenta principalmente para que nesse momento incial eu apenas me preocupe com o texto e sua organização, e ainda, por ser apenas texto, facilita sua distribuição e versionamento.

Ilustrações? No Defensores Beta ainda não tem, simplesmente porque eu não sei desenhar! Ha ha! Se quiser usar os canais de comunicação do Defensores para sugerir ilustrações livres em preto e branco para versão final do Defensores ficarei grato. 

Versão impressa? O Latex gera um arquivo em pdf pronto para impressão, quando tiver a versão final do Defensores distribuirei da mesma maneira. Prometo também que estudarei as possiblidades com alguma editora de ter uma versão de impressão legal, amigável e divertida igual a do manual 3d\&T.

Espero que você se integre a comunidade dos Defensores e fique por dentro das novidades que surgiram. Abraços!

